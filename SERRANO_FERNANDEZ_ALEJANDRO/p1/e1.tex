Para el caso del centro de extracción de materiales he seguido la siguiente estrategia. Aquellas celdas que se encuentren cerca del obstáculo con mayor radio, 
tendrán un mayor valor. En cambio, si estos obstáculos se encuentran cerca del borde del mapa, se escogerá aquel obstáculo que más al centro del mapa se encuentre.
Para determinar si un obstáculo está cerca del borde del mapa, supondré un margen de un 40\% desde los extremos del mapa. Por ejemplo, si el mapa tiene una anchura y un
alto de 340, todas los obstáculos fuera del margen de un ancho y un alto de 204 se descartarán, y se valorarán aquellos obstáculos que más al centro se encuentren.


En cuanto al diseño de la función, vamos tomando todos los obstáculos y almacenaremos aquellos que tienen mayor radio y aquellos que más al centro del mapa se encuentren.
Para determinar el valor del obstáculo que más al centro se encuentra, partimos de una variable valor inicializada a 255 (el valor máximo definido por mí), y conforme más 
alejado esté el obstáculo del centro de mapa, se le irá restando valor.

Una vez determinado los obstáculos candidatos, comprobaremos si el obstáculo con mayor radio se encuentra dentro del margen, tal y como expliqué anteriormente. En caso contrario escogeremos
el más cercano al centro del mapa.

Finalmente repetimos la misma estrategia anterior para determinar los valores de las celdas más cercanas al obstáculo seleccionado. Inciamos de nuevo la variable valor a 255 y vamos restándole
valor conforme más alejadas estén del obstáculo seleccionado.
%Elimine los símbolos de tanto por ciento para descomentar las siguientes instrucciones e incluir una imagen en su respuesta. La mejor ubicación de la imagen será determinada por el compilador de Latex. No tiene por qué situarse a continuación en el fichero en formato pdf resultante.
%\begin{figure}
%\centering
%\includegraphics[width=0.7\linewidth]{./defenseValueCellsHead} % no es necesario especificar la extensión del archivo que contiene la imagen
%\caption{Estrategia devoradora para la mina}
%\label{fig:defenseValueCellsHead}
%\end{figure}