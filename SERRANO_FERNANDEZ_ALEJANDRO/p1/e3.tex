\begin{lstlisting}
    float cellWidth = mapWidth / nCellsWidth;
    float cellHeight = mapHeight / nCellsHeight; 
    bool colocado = false;
 
    //Tomamos la primera defensa
    List<Defense*>::iterator currentDefense = defenses.begin();
    std::vector<Celda> celdasCandidatas;
    
    //Introducimos todas las celdas candidatas, con sus respectivos valores
    for(int i = 0; i < nCellsWidth; i++)
    {
    	for(int j = 0; j < nCellsHeight; j++)
    	{
    		celdasCandidatas.push_back(Celda(i,j,cellValue(i, j, freeCells, nCellsWidth, nCellsHeight, mapWidth, mapHeight, obstacles, defenses)));
    	}
    }
    
    //Ordenamos el vector, para asi obtener primero las celdas con mayor valor
    sort(celdasCandidatas.begin(),celdasCandidatas.end());    
    
    //Colocamos el centro de extraccion en la celda factible con mayor valor
    while(!celdasCandidatas.empty() && !colocado)
    {
    	if(factible(celdasCandidatas.back(), freeCells, nCellsWidth, nCellsHeight, mapWidth, mapHeight, obstacles, defenses, defenses.begin()))
    	{
    		(*currentDefense)->position.x = (celdasCandidatas.back().row() * cellWidth) +  (0.5f * cellWidth);
    		(*currentDefense)->position.y = (celdasCandidatas.back().col() * cellHeight) + (cellHeight * 0.5f);
    		colocado = true;
    	}    	
    	celdasCandidatas.pop_back();
    }
\end{lstlisting}