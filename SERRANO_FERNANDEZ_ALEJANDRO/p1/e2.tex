La función de factibidad realizada recibirá una celda y será evaluda de tal manera que cumpla los siguientes requisitos:


1) Que no se interponga con ningún obstáculo. Para ello comprobaremos que la suma del radio de la defensa a colocar y de la defensa a comparar, no sea mayor que la distancia entre ambos centros.


2) Que no se interponga con ninguna otra defensa. Para ello comprobaremos de tal manera que el apartado anterior.


3) Que no salga del mapa, comprobando que la suma del radio y el centro de la defensa a colocar no sobrepase los límites del mapa.


En cuanto a su implementación, ésta función recorrerá todos los obstáculos y defensas, y determinará si la celda escogida se interpone con alguna de ellas. Para ello sumaremos el radio de la 
defensa a colocar y el radio de la defensa u obstáculo a comparar. Si la suma de los radios es mayor que la distancia entre ambos centros, daremos por concluido que celda no es válida. Para determinar la 
distancia entre ambos centros, hemos de definir previamente una función distancia() que realizará dicho cálculo a través del teorema de Pitágoras.

Finalmente, la función determinará si la defensa no sale del mapa, para ello sumamos el radio de la defensa y el centro de la celda a comprobar. Si la suma supera los límites del mapa, entonces descartaremos dicha celda.
\hfill \break
\hfill \break
\hfill \break





